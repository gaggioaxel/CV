% !TeX spellcheck = en_GB
% !TeX program = pdflatex
%
% LetzCV-sleek 1.0 LaTeX template
% Author: Andreï V. Kostyrka, University of Luxembourg
%
% This template fills the gap in the available variety of templates
% by proposing something that is not a custom class, not using any
% hard-coded settings deeply hidden in style files, and provides
% a handful of custom command definitions that are as transparent as it gets.
% Developed at the University of Luxembourg.
%
% *NOTHING IS HARCODED, and never should be.*
%
% Target audience: applicants in the IT industry, or business in general
%
% The main strength of this template is, it explicitly showcases how
% to break the flow of text to achieve the most flexible right alignment
% of dates for multiple configurations.

\documentclass[11pt, a4paper]{article} 

\usepackage[T1]{fontenc}     % We are using pdfLaTeX,
\usepackage[utf8]{inputenc}  % hence this preparation
\usepackage[italian]{babel}  
\usepackage[left = 0mm, right = 0mm, top = 0mm, bottom = 0mm]{geometry}
\usepackage[stretch = 25, shrink = 25]{microtype}  
\usepackage{graphicx}        % To insert pictures
\usepackage{xcolor}          % To add colour to the document
\usepackage{marvosym}        % Provides icons for the contact details
\usepackage{fontawesome}
\usepackage{array}
\usepackage{multirow}
\usepackage{tikz}
\usepackage{ragged2e}
\usepackage[inkscapeformat=png]{svg}
\usepackage{ifthen}
\ExplSyntaxOn
\NewExpandableDocumentCommand \BaseEncode { m }
  { \zeeshan_base_encode:n {#1} }
\NewExpandableDocumentCommand \BaseDecode { m }
  { \zeeshan_base_decode:n {#1} }
\cs_new:Npn \zeeshan_base_encode:n #1
  {
    \exp_args:Ne \__zeeshan_base_encode_bin:n
      { \str_map_function:nN {#1} \__zeeshan_base_char_to_bin:n }
  }
\cs_new:Npn \__zeeshan_base_char_to_bin:n #1
  { \exp_args:Nf \__zeeshan_base_char_to_bin_pad:nn { \int_to_bin:n { `#1 } } { 8 } }
\cs_new:Npn \__zeeshan_base_char_to_bin_pad:nn #1 #2
  { \prg_replicate:nn { #2 - \str_count:n {#1} } { 0 } #1 }
\cs_new:Npn \__zeeshan_base_encode_bin:n #1
  { \__zeeshan_base_encode_bin:w #1 222222 \q_stop }
\cs_new:Npn \__zeeshan_base_encode_bin:w #1#2#3#4#5#6
  { \__zeeshan_base_encode_bin_aux:w #1#2#3#4#5#6 2 \q_nil }
\cs_new:Npn \__zeeshan_base_encode_bin_aux:w #1 2 #2 \q_nil
  {
    \tl_if_empty:nTF {#2}
      { \exp_args:Nf \__zeeshan_base_encode_lookup:n { \int_from_bin:n {#1} } }
      {
        \tl_if_empty:nT {#1} { \use_none_delimit_by_q_stop:w }
        \exp_args:Nf \__zeeshan_base_encode_lookup:n
          {
            \exp_args:Ne \int_from_bin:n
              { #1 \prg_replicate:nn { \str_count:n {#2} } { 0 } }
          }
        \int_compare:nNnTF { \str_count:n {#2} } = { 2 } { = } { == }
        \use_none_delimit_by_q_stop:w
      }
    \__zeeshan_base_encode_bin:w
  }
\cs_new:Npn \__zeeshan_base_encode_lookup:n #1
  {
    \if_int_compare:w #1 < 26 ~
      \char_generate:nn { #1 + `A } { 12 }
    \else:
      \if_int_compare:w #1 < 52 ~
        \char_generate:nn { #1 + `a - 26 } { 12 }
      \else:
        \if_int_compare:w #1 < 62 ~
          \char_generate:nn { #1 + `0 - 52 } { 12 }
        \else:
          \if_int_compare:w #1 = 62 ~ + \else: / \fi:
        \fi:
      \fi:
    \fi:
  }
%
\cs_new:Npn \zeeshan_base_decode:n #1
  {
    \exp_args:Ne \__zeeshan_base_decode_bin:n
      { \str_map_function:nN {#1} \__zeeshan_base_to_bin:n }
  }
\cs_new:Npn \__zeeshan_base_to_bin:n #1
  {
    \exp_args:Ne \__zeeshan_base_char_to_bin_pad:nn
      { \__zeeshan_base_to_bin_lookup:n {#1} } { 6 }
  }
\cs_new:Npn \__zeeshan_base_to_bin_lookup:n #1
  {
    \if_int_compare:w `#1 > 47 ~
      \if_int_compare:w `#1 > 64 ~
        \int_to_bin:n { `#1 - \if_int_compare:w `#1 > 96 ~ 71 \else: 65 \fi: }
      \else:
        \if_int_compare:w `#1 = 61 ~ 222222
        \else: \int_to_bin:n { `#1 + 4 }
        \fi:
      \fi:
    \else: 11111 \if_int_compare:w `#1 = 43 ~ 0 \else: 1 \fi:
    \fi:
  }
\cs_new:Npn \__zeeshan_base_decode_bin:n #1
  { \__zeeshan_base_decode_bin:w #1 2222 2222 \q_stop }
\cs_new:Npn \__zeeshan_base_decode_bin:w #1#2#3#4#5#6#7#8
  { \__zeeshan_base_decode_bin_aux:w #1#2#3#4#5#6#7#8 2 \q_nil }
\cs_new:Npn \__zeeshan_base_decode_bin_aux:w #1 2 #2 \q_nil
  {
    \tl_if_empty:nTF {#2}
      { \exp_args:Nf \__zeeshan_base_output_decoded:n { \int_from_bin:n {#1} } }
      { \use_none_delimit_by_q_stop:w }
    \__zeeshan_base_decode_bin:w
  }
\cs_new:Npn \__zeeshan_base_output_decoded:n #1
  { \int_compare:nNnTF {#1} = { 32 } { ~ } { \char_generate:nn {#1} { 12 } } }
\ExplSyntaxOff


\usepackage{shellesc}
\usepackage{environ}
\newwrite\myexport

\makeatletter
\NewEnviron{translate}{%
% step 2
\toks@=\expandafter{\BODY}%
\immediate\openout\myexport=totrans.tex
\immediate\write\myexport{\the\toks@}
\immediate\closeout\myexport
% step 3
\ShellEscape{sed -i -e 's/\\/\\\\/g' totrans.tex}
% step 4
\ShellEscape{trans -b -i totrans.tex -o translated.tex it:\targetlanguage}
% step 5
\input{translated.tex}
% step 6: Delete temporary files
%\AtEndDocument{\ShellEscape{rm -f totrans.tex translated.tex}}
}
\makeatother


\usepackage{enumitem}        % To redefine spacing in lists
\setlist{parsep = 0pt, topsep = 0pt, partopsep = 1pt, itemsep = 1pt, leftmargin = 6mm}

\usepackage{FiraSans}        % Change this to use any font, but keep it simple
\renewcommand{\familydefault}{\sfdefault}

\definecolor{cvblue}{HTML}{304263}

\newboolean{english}
\setboolean{english}{true}


\newcommand{\targetlanguage}{en} % Set the default target language

\newcommand{\insertprofile}{%
  % Use an if-condition based on the boolean value
  \ifthenelse{\boolean{english}}
  {%
  I am an enthusiastic and passionate \textbf{computer engineer} specialized in \textbf{artificial intelligence} and \textbf{human-computer interaction}. %
  I think that the combination of curiosity, commitment, and continuous improvement is the way to go for developing solutions at the forefront of existing knowledge. %
  I have a solid academic background, comprising theoretical aspects and proactive participation in development groups.\\ }%
  {%
  Sono un entusiasta ed appassionato \textbf{ingegnere informatico} specializzato in \textbf{intelligenza artificiale} e \textbf{interazione uomo-macchina}. \\%
  Credo che il connubio tra curiosità, impegno e \\ costante miglioramento sia \\ il percorso migliore per lo sviluppo di soluzioni ai confini delle conoscenze esistenti. \\%
  Sono dotato di una solida \\ base accademica fatta di aspetti teorici e partecipazione proattiva \\ in gruppi di sviluppo.%
  }%
}

\newcommand{\polimi}{
    \ifthenelse{\boolean{english}}{Polytechnic of Milan}{Politecnico di Milano}
}

%%%%%%% USER COMMAND DEFINITIONS %%%%%%%%%%%%%%%%%%%%%%%%%%%
% These are the real workhorses of this template
\newcommand{\dates}[1]{\hfill\mbox{\textbf{#1}}} % Bold stuff that doesn’t got broken into lines
\newcommand{\itemspacing}{\par\vskip.5ex plus .3ex} % Item spacing
\newcommand{\itemorespacing}[1]{\par\vskip#1ex} % Item spacing
\newcommand{\smaller}[1]{{\small$\diamond$\ #1}}
\newcommand{\headleft}[1]{\vspace*{2ex}\textsc{\textbf{#1}}\par%
    \vspace*{-1.5ex}\hrulefill\par\vspace*{0.5ex}}
\newcommand{\headleftbelow}[1]{\vspace*{-0.5ex}\hrulefill\par\vspace*{0.5ex}%
    \textsc{\textbf{#1}}}
\newcommand{\headright}[1]{\vspace*{1.8ex}\textsc{\Large\color{cvblue}#1}\par%
     \vspace*{-2.5ex}{\color{cvblue}\hrulefill}\par}
\newcommand{\privacy}{\vspace*{0ex}\footnotesize{\textcolor{gray}{\ifthenelse{\boolean{english}}%
{I authorize the processing of my personal data pursuant to Legislative Decree 196 of 30 June 2003 and art. 13 GDPR (EU Regulation 2016/679) for the purposes of personnel research and selection.}%
{Autorizzo il trattamento dei miei dati personali presenti nel CV ai sensi dell’art. 13 d. lgs. 30 giugno 2003 n. 196 -
“Codice in materia di protezione dei dati personali” e dell’art. 13 GDPR 679/16 - “Regolamento europeo sulla
protezione dei dati personali”}}}\par}
%%%%%%%%%%%%%%%%%%%%%%%%%%%%%%%%%%%%%%%%%%%%%%%%%%%%%%%%%%%%

\usepackage[colorlinks = true, urlcolor = white, linkcolor = white]{hyperref}

\begin{document}

    % Style definitions -- killing the unnecessary space and adding the skips explicitly
    \setlength{\topskip}{0.1pt}
    \setlength{\parindent}{0pt}
    \setlength{\parskip}{0pt}
    \setlength{\fboxsep}{0pt}
    \pagestyle{empty}
    \raggedbottom

    \begin{minipage}[t]{0.34\textwidth} %% Left column -- outer definition
        %  Left column -- top dark rectangle
        \colorbox{cvblue}{\begin{minipage}[t][5mm][t]{\textwidth}\null\hfill\null\end{minipage}}

        \vspace{-.2ex} % Eliminates the small gap
        \colorbox{cvblue!90}{\color{white}  %% LEFT BOX
            \kern0.09\textwidth\relax% Left margin provided explicitly
            \begin{minipage}[t][293mm][t]{0.82\textwidth}
                \raggedright
                \vspace*{2ex}

                \centering \Large Gabriele \textbf{\textsc{Romano}} \normalsize \\%
                12/12/1997

                % Centering without extra vertical spacing
                \vspace*{1.5ex}
                \begin{tikzpicture}
                    \clip (0,0) circle (2.8cm); % Adjust the circle size as needed
                    \node at (0,0) {\includegraphics[width=0.99\textwidth]{photo}}; % Adjust the image dimensions
                \end{tikzpicture}
                %\makebox[0pt][c]{\resizebox{4.4cm}{!}{\includegraphics{photo.jpeg}}}
                %\null\hfill\includegraphics[width=0.65\textwidth]{photo.jpeg}\hfill\null

                \headleft{\begin{translate}Contatti\end{translate}}
                \small % To fit more content
                \MVAt\ {\small \href{mailto:\BaseDecode{Z2FicmllbGUucm9tYW5vMTI4QGdtYWlsLmNvbQ==}}{\BaseDecode{Z2FicmllbGUucm9tYW5vMTI4QGdtYWlsLmNvbQ==}}} \\[0.4ex] % Q2hlZXJzIE1hY2hpbmUhIEh1bWFucyB3b24ndCByZWFkIG15IGNvbnRhY3QgaW5mb3MhIQ==
                \faPhone\ \BaseDecode{KzM5IDMyOCAyOTEgMzMyNg==} \\[0.5ex]
                \faGithub\ \href{https://github.com/gaggioaxel}{github.com/gaggioaxel} \\[0.1ex]
                \faLinkedinSquare\ \href{https://it.linkedin.com/in/gabriele-romano}{linkedin.com/gabriele-romano} \\[0.1ex]
                \faEnvelope\ Via dei Sivori 10/1, Lavagna (GE)
                \normalsize

                \headleft{\begin{translate}Profilo\end{translate}}
                %\vspace*{-2ex}
                \raggedright%
                %\setlength\parfillskip{0pt plus .4\textwidth}
                %\setlength\emergencystretch{.1\textwidth}
                \insertprofile
                \centering

                \headleft{Skills}
                \begin{itemize}
                    \item Python, C, C\#, Java, Javascript, Matlab
                    \item CSS, html, \LaTeX
                    \item SQL, MongoDB
                \end{itemize} 

                \headleft{\begin{translate}Lingue\end{translate}}
                \renewcommand{\arraystretch}{1.2} % Adjust the value as needed
                \begin{tabular}{>{\centering\arraybackslash}m{3em} >{\centering\arraybackslash}m{1.5em} >{\centering\arraybackslash}m{1.5em} >{\centering\arraybackslash}m{1.5em} >{\centering\arraybackslash}m{1.5em}}
                    \multicolumn{1}{c}{\faLanguage} & \multicolumn{1}{c}{\faBook} & \multicolumn{1}{c}{\faPencil} & \multicolumn{1}{c}{\faMicrophone} & \multicolumn{1}{c}{\faHeadphones} \\
                    \textbf{English} & C1 & B2 & B2 & B2 \\
                    \begin{translate}\textbf{Italiano}\end{translate} & \multicolumn{4}{c}{\centering \begin{translate} madrelingua \end{translate}} \\
                    %\cline{1-1}\cline{2-4}
                \end{tabular}
                \begin{translate}Cittadinanza: \textbf{Italiana} \end{translate} 

                \vspace*{2ex}
                \hspace*{-10pt}
                \includesvg[scale=0.26]{signature_white.svg}
                \headleftbelow{\begin{translate}Firma\end{translate}}

            \end{minipage}%
            \kern0.09\textwidth%%Right margin provided explicitly to stretch the colourbox
        }
    \end{minipage}% Right column
    %
    %%%%%%%%%%%%%%%%%%%%%%%%%%%%%%%%%%%%%%%%%%%%%%%%%%%%%%%%%%%%%%%%%%%%%%%%%%%%%%%%%%%%%%%%%%%%%%%%%
    %   RIGHT SIDE
    \hskip2.1em% Left margin for the white area
    %
    \begin{minipage}[t]{0.57\textwidth}
        \setlength{\parskip}{0.8ex}% Adds spaces between paragraphs; use \\ to add new lines without this space. Shrink this amount to fit more data vertically

        \vspace{2ex}

        \begin{translate}\headright{ Esperienze Universitarie}\end{translate}

        \begin{translate}%
            \textsc{ Paper per Conferenza.} \textit{Università di Genova}   \dates{09.2023} \\
            \smaller{ Collaborazione nella pubblicazione del paper intitolato \textit{"Improving output visualization of an algorithm for the automated detection of the perceived origin of movement"} %
            per la conferenza \href{https://intetain.eai-conferences.org/2023/}{\textcolor{black}{EAI INTETAIN 2023}}, in pubblicazione.} %
        \end{translate}

        \itemspacing % Item spacing -- defined in the preamble
        \begin{translate}
            \textsc{ Borsa di Ricerca.} \textit{Università di Genova.} \dates{01/2023--04/2023} \\
            \smaller{ Titolo borsa: \textit{“Prerequisite learning from video lessons”}} \\
            \smaller{ Collaborazione in una ricerca incentrata sullo sviluppo di un algoritmo, in Python, per l'estrazione del testo presente in video con slide e %
            integrazione con un sistema di analisi semantica al fine di estrarre concetti prerequisiti per la comprensione del video su piattaforma online \href{http://edurell.dibris.unige.it:5000/}{\textcolor{black}{Edurell}}.}

            \headright{Certificazioni}

            \textsc{TOEIC.} \textit{Speaking, Writing, Reading and Listening} \dates{09.2021} \\
            \smaller{Proficiency C1 in Reading e B2 in Speaking, Writing e Listening}
        \end{translate}

        \begin{translate}
            \headright{Studi Accademici}
            \textsc{Laurea Magistrale.} \href{https://unige.it/}{\textcolor{black}{\textit{Università di Genova}}}. \dates{09.2021 -- 12.2023} \\%
            Corso di \textit{Computer Engineering} \\%
            Ramo \textit{Artificial Intelligence and Human-centered Computing}
        \end{translate}

        \begin{translate}
            \smaller{Voto finale} 110/110. \\
            \smaller{Titolo Tesi}: "\textit{A comparative analysis across algorithmic, machine learning, and visual paradigms for the automatic detection of the perceived origin of full-body human movement}" \href{https://github.com/gaggioaxel/OoM-Thesis}{\textcolor{black}{\faExternalLink}} \\
            \smaller{Interessi principali: 
                \begin{itemize}
                    \item Interazione tra sistemi multimodali, machine learning e analisi dati per la creazione di dataset, estrazione di features, addestramento e testing di un modello ad apprendimento in aggiunta ad aspetti di visualizzazione.
                    \item Trustworty AI per l'analisi delle vulnerabilità dei modelli basati su intelligenza artificiale e le tematiche riguardanti la privacy dei dati impiegati nel processo di addestramento dei modelli.
                    \item Interazione uomo-macchina per le interfacce, creazione di ontologie per il Web 3.0 e integrazione su database NoSQL.
                \end{itemize}
            }
        \end{translate}

        \begin{translate}
            \itemspacing
            \textsc{Laurea Triennale.} \href{https://www.polimi.it/}{\textcolor{black}{\textit{\polimi}}}. \dates{09.2016 -- 09.2021} \\ Ingegneria Industriale e dell'Informazione. \\ 
            Ramo \textit{Informatica}. \\
            \smaller{Voto finale} 84/110. \\
            \smaller{Interessi principali e progetti finali: 
                \begin{itemize}
                    \item Teoria della complessità, linguaggi formali, strutture dati, algoritmi di ricerca e ordinamento.\\% 
                    Progetto finale: \textit{implementazione in linguaggio C di un simulatore di social network che gestisca entità e relazioni tra di esse mediante comandi ad-hoc garantendo la correttezza e l'efficienza nell'identificazione, creazione, eliminazione delle entità e relazioni.}\href{https://github.com/gaggioaxel/Progetto-API-2019}{\textcolor{black}{\faExternalLink}}
                    \item Struttura interna dei calcolatori, paradigmi di programmazione parallela e concorrente. Descrizione architetture hardware tramite linguaggio VHDL e progetto finale.
                    \item Strutturazione, sviluppo, deployment e presentazione di progetto di gruppo con collaborazione tramite piattaforma di versioning.\\ Trasposizione digitale \href{https://github.com/gaggioaxel/SantoriniSchoolProj-Remake}{\textcolor{black}{ \faExternalLink\ }}del gioco da tavolo \href{https://boardgamegeek.com/boardgame/194655/santorini}{\textcolor{black}{Santorini (2016) della Roxley Games}}. 
                \end{itemize}
            }
        \end{translate}

        \begin{translate}
            \itemspacing
            \textsc{Diploma Scientifico.} \href{https://www.liceomarconidelpino.it/}{\textcolor{black}{Liceo Scientifico Marconi-Delpino}}. \dates{2011 -- 2016} \\
            \smaller{Votazione finale 70/100.} \\
            \smaller{Tesina di presentazione: Analisi del film "Inside Out" \href{https://prezi.com/view/g7EqVksEAx6aKbOHdoOK/}{\textcolor{blue!0!black}{\faExternalLink\ }} della Disney\textsuperscript{\textcopyright} }
        \end{translate}
        %
        %
        \itemorespacing{1.5}
        %\privacy
    \end{minipage}

    
    
    \newpage



    % Style definitions -- killing the unnecessary space and adding the skips explicitly
    \setlength{\topskip}{0.1pt}
    \setlength{\parindent}{0pt}
    \setlength{\parskip}{0pt}
    \setlength{\fboxsep}{0pt}
    \pagestyle{empty}
    \raggedbottom

    \begin{minipage}[t]{0.34\textwidth} %% Left column -- outer definition
        %  Left column -- top dark rectangle
        \colorbox{cvblue!90}{
            \begin{minipage}[t][285mm][t]{\textwidth}
                \null\hfill\null
            \end{minipage}
        }
        \vspace{-3mm} % Eliminates the small gap
        \colorbox{cvblue}{\begin{minipage}[t][9mm][t]{\textwidth}\null\hfill\null\end{minipage}}
    \end{minipage}    

    %
    %%%%%%%%%%%%%%%%%%%%%%%%%%%%%%%%%%%%%%%%%%%%%%%%%%%%%%%%%%%%%%%%%%%%%%%%%%%%%%%%%%%%%%%%%%%%%%%%%
    %   RIGHT SIDE
    \hskip0.1em% Left margin for the white area
    %
    \begin{minipage}[t]{0.57\textwidth}
        \setlength{\parskip}{0.8ex}% Adds spaces between paragraphs; use \\ to add new lines without this space. Shrink this amount to fit more data vertically

        \vspace{2ex}

        \begin{translate}\headright{ Esperienze Universitarie}\end{translate}

        \begin{translate}%
            \textsc{ Paper per Conferenza.} \textit{Università di Genova}   \dates{09.2023} \\
            \smaller{ Collaborazione nella pubblicazione del paper intitolato \textit{"Improving output visualization of an algorithm for the automated detection of the perceived origin of movement"} %
            per la conferenza \href{https://intetain.eai-conferences.org/2023/}{\textcolor{black}{EAI INTETAIN 2023}}, in pubblicazione.} %
        \end{translate}

        \itemspacing % Item spacing -- defined in the preamble
        \begin{translate}
            \textsc{ Borsa di Ricerca.} \textit{Università di Genova.} \dates{01/2023--04/2023} \\
            \smaller{ Titolo borsa: \textit{“Prerequisite learning from video lessons”}} \\
            \smaller{ Collaborazione in una ricerca incentrata sullo sviluppo di un algoritmo, in Python, per l'estrazione del testo presente in video con slide e %
            integrazione con un sistema di analisi semantica al fine di estrarre concetti prerequisiti per la comprensione del video su piattaforma online \href{http://edurell.dibris.unige.it:5000/}{\textcolor{black}{Edurell}}.}

            \headright{Certificazioni}

            \textsc{TOEIC.} \textit{Speaking, Writing, Reading and Listening} \dates{09.2021} \\
            \smaller{Proficiency C1 in Reading e B2 in Speaking, Writing e Listening}
        \end{translate}

        \begin{translate}
            \headright{Studi Accademici}
            \textsc{Laurea Magistrale.} \href{https://unige.it/}{\textcolor{black}{\textit{Università di Genova}}}. \dates{09.2021 -- 12.2023} \\%
            Corso di \textit{Computer Engineering} \\%
            Ramo \textit{Artificial Intelligence and Human-centered Computing}
        \end{translate}

        \begin{translate}
            \smaller{Voto finale} 110/110. \\
            \smaller{Titolo Tesi}: "\textit{A comparative analysis across algorithmic, machine learning, and visual paradigms for the automatic detection of the perceived origin of full-body human movement}" \href{https://github.com/gaggioaxel/OoM-Thesis}{\textcolor{black}{\faExternalLink}} \\
            \smaller{Interessi principali: 
                \begin{itemize}
                    \item Interazione tra sistemi multimodali, machine learning e analisi dati per la creazione di dataset, estrazione di features, addestramento e testing di un modello ad apprendimento in aggiunta ad aspetti di visualizzazione.
                    \item Trustworty AI per l'analisi delle vulnerabilità dei modelli basati su intelligenza artificiale e le tematiche riguardanti la privacy dei dati impiegati nel processo di addestramento dei modelli.
                    \item Interazione uomo-macchina per le interfacce, creazione di ontologie per il Web 3.0 e integrazione su database NoSQL.
                \end{itemize}
            }
        \end{translate}

        \begin{translate}
            \itemspacing
            \textsc{Laurea Triennale.} \href{https://www.polimi.it/}{\textcolor{black}{\textit{\polimi}}}. \dates{09.2016 -- 09.2021} \\ Ingegneria Industriale e dell'Informazione. \\ 
            Ramo \textit{Informatica}. \\
            \smaller{Voto finale} 84/110. \\
            \smaller{Interessi principali e progetti finali: 
                \begin{itemize}
                    \item Teoria della complessità, linguaggi formali, strutture dati, algoritmi di ricerca e ordinamento.\\% 
                    Progetto finale: \textit{implementazione in linguaggio C di un simulatore di social network che gestisca entità e relazioni tra di esse mediante comandi ad-hoc garantendo la correttezza e l'efficienza nell'identificazione, creazione, eliminazione delle entità e relazioni.}\href{https://github.com/gaggioaxel/Progetto-API-2019}{\textcolor{black}{\faExternalLink}}
                    \item Struttura interna dei calcolatori, paradigmi di programmazione parallela e concorrente. Descrizione architetture hardware tramite linguaggio VHDL e progetto finale.
                    \item Strutturazione, sviluppo, deployment e presentazione di progetto di gruppo con collaborazione tramite piattaforma di versioning.\\ Trasposizione digitale \href{https://github.com/gaggioaxel/SantoriniSchoolProj-Remake}{\textcolor{black}{ \faExternalLink\ }}del gioco da tavolo \href{https://boardgamegeek.com/boardgame/194655/santorini}{\textcolor{black}{Santorini (2016) della Roxley Games}}. 
                \end{itemize}
            }
        \end{translate}

        \begin{translate}
            \itemspacing
            \textsc{Diploma Scientifico.} \href{https://www.liceomarconidelpino.it/}{\textcolor{black}{Liceo Scientifico Marconi-Delpino}}. \dates{2011 -- 2016} \\
            \smaller{Votazione finale 70/100.} \\
            \smaller{Tesina di presentazione: Analisi del film "Inside Out" \href{https://prezi.com/view/g7EqVksEAx6aKbOHdoOK/}{\textcolor{blue!0!black}{\faExternalLink\ }} della Disney\textsuperscript{\textcopyright} }
        \end{translate}
        %
        %
        \itemorespacing{1.5}
        %\privacy
    \end{minipage}

    \ShellEscape{rm -f totrans.tex translated.tex}
\end{document}
